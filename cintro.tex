\section{Conclusion }
\label{ch:cintro}

Afin de pouvoir être lu et compris par un maximum de personne, j'ai décidé au travers de cette introduction de poser les bases d'une application d'entreprises ainsi que de tout les concepts important entourant JavaScript et de ses avancés.

Je vais utilise le modèle trois-tiers pour expliquer comment JavaScript intervient dans chaque couche pour développer l'application avec ce langage. La présentation de l'histoire permet de comprendre pourquoi JavaScript est si peu répandu ailleurs que dans le navigateur, Le fait que JavaScript est un processus de standardisation prouve que ce n'est pas un langage écrit il y'a longtemps et n'évoluant pas. Au contraire c'est un langage avec un processus d'évolution et de standardisation, à tel point que la futur version de JavaScript promet de combler certaines critique fait à son encontre. Il est également important de comprendre l'intêret du JSON comme format d'échange standard . L'architecture d'AJAX à permis à JavaScript de réssucité et de succité également énormément d'intêret. jQuery est arrivé et à combler les problème d'interopérabilité entre les navigateurs, ce qui à permis de fédérer une communauté autour du produit mais également autour de JavaScript. Le Restfull à également été une révolution dans le Web en permettant de faire des échanges entre serveurs et clients grâce à des API public en se basant uniquement sur le protocole simple et reconnu HTTP. J'ai également décidé de présenter le modèle de conception MVC afin de mieux comprendre pourquoi les frameworks inclus ce schéma par défaut dans le développement d'application Node.js ou coté client.

J'espère qu'après la lecture de mon mémoire, vous serez et comprendrez l'enthousiasme entourant actuellement JavaScript et Node.js Et en quoi sont arrivé sur le marché remet en question le fameux modèle LAMP (Linux, Apache, MySQL et PHP) et même la conception d'application traditionnelle.


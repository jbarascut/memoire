\chapter{Ajax}
\label{ch:ajax}

\section*{Introduction}

AJAX (Asynchronous JavaScript And XML) est une architecture informatique permettant de construire des sites web dynamiques interactifs et des applications Web en se servant de différentes technologies présenté précedment.

Ajax combine JavaScript, XML, CSS, le DOM et XMLHttpRequest afin d’améliorer la maniabilité et le confort d’utilisation des Applications Internet Riches (RIA).

Le terme Ajax a été introduit par Jesse James Garrett, le 18 février 2005, dans un article sur le site Web Adaptive Path. Ajax a créé une petite révolution dans les navigateurs.

En utilisant Ajax, le diaglogue entre le navigateur et le serveur se déroule la plupart du temps de la manière suivante:

 
Graphique techno ajax

Les échanges de données entre le client et le serveur peuvent utiliser d’autres formats notamment JSON.

Ajax à permi a JavaScript de gagner en popularité et de ne plus être le vilain petit canard des langages de programmation. Ainsi avec AJAX javaScript n’est plus une langage afichant des popup intempestives et vérifiant des formulaires.

Google a marqué les esprits avec Google Maps. Maps n’aurait jamais pu éxister sans Ajax. L’utilisation de JavaScript par Google a permis a JS de montrer son potentiel. Depuis JS s’est désinhibé, et de véritables logiciels sont apparus dans nos navigateurs.

Google est un très gros consomateurs de JS notament avec Drive ou Gmail.

JavaScript est donc passé du petit langage d’agrément pour pages web au langage de développement d’applications réseau supporté par tout les navigateurs quelques soit le système d’exploitation. Et le navigateur, pour un grand nombre d’utilisateurs, est la porte d’entrée de l’ordinateur et du réseau.

Avant AJAX: la page web est le support pour de petites applications Js, le plus souvent d’agrément et facultatives pour utiliser le contenu.

Avec AJAX: Js deviens le coeur du site. Il génère du contenu HTML dont il a la maitrise. Avec un autre langage coté serveur, Js coté client s’appuie sur le moteur graphique du navigateur pour générer l’interface de l’application, plus pratique que n’importe quel “toolkit”.

Cette mutation a pris du temps. Le navigateur est passé d’un logiciel pour consulter des pages web à un logiciel permettant d’exécuter d’autres logiciels (en somme comme un système d’exploitation).

Ajax est donc plébiscité parce qu’il donne naissance à des applications comme jamais nous avions vu par la fenere du navigateur. Avec AJAX une application comme Google Earth pourrait aussi bien exister de façon autonome.

Mais à cet époque, JavaScript à encore des inconvénient majeur empechant son adoption en masse.

JavaScript étant exécuter par le navigateur les performances à l’époque était extremement faible. Acuellement la bataille des navigateurs se au meilleurs support de HTML5/CSS3 mais aussi sur les performances des moteurs JS. L’arrivé de Google Chrome et des énormes besoins de Google en performance JavaScript ont poussé tout les acteurs du marché à améliorer les moteurs pour toujours plus de performances.

Le second problème de JS est l’éxécution du script par le moteur. Suivant le navigateur le code js doit être différent, chaque éditeur étant libre d’integrer JS comme il le souhaite. Ainsi un code js peut fonctionner correctement sur Firefox ou Chrome mais pas sur IE.

Le Javascript est aussi un langage qui nécessite des efforts importants et des développements en AJAX pur peuvent être extrêmement coûteux.

Afin de résoudre se problème de développement  John Resig à sortie un framework révolutionnant la façon d’écrire du code JavaScript. Ce framework: Jquery.


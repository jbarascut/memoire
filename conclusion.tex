\chapter{Conclusion}

À travers ce mémoire, nous avons pu voir que JavaScript a un très fort potentiel dans le développement 3-tiers d'application multi-plateformes. Nous avons abordé les principaux composant de l'architecture 3-tiers et de comment JavaScript peut répondre à chaque composant. Ainsi nous avons pu voir que JavaScript à une réponse pour chaque problématique.

Le JavaScript,a de nombreux atouts, notamment son interopérabilité sur l'ensemble des navigateurs récents du marché et sa bataille importante pour toujours plus de performance et de fonctionnalités mené par Google Chrome et Mozilla Firefox.


JavaScript a actuellement un regain d'actualité avec l'explosion du marché des smartphones et la multiplication des navigateurs mobiles et des systèmes d'exploitation. De plus l'arrivé de nouveaux concurrents comme Tizen, Ubuntu Touch et Firefox OS va stimuler la concurrence autour du développement d'application multi-plateformes.

Le JavaScript avec Node.js s'invite du coté du serveur et répond à une forte demande concernant les applications temps réel et à forte demande. Node.js possède un nombre impressionnant de module et est un des dépôts les plus suivis sur github. La feuille de route de Node.js pour la version 1.0 est impressionnante ce qui reste prometteur pour son avenir.

Les développeurs du libre ont particulièrement bien compris ce nouveau besoin. Si l’on regarde les statistiques de Github, JavaScript représente 21\% des projets hébergés. Ce qui montre la tendance du libre à utilisé massivement JavaScript dans l’avenir.

Les systèmes d'exploitations ont également bien saisi cette tendance en misant également sur le JavaScript pour le développement de leur application. Ainsi Microsoft inclus le support de JavaScript dans Windows 8 et à ainsi introduit le premier langage non développer par la firme.

Néanmoins tout n'est pas forcément au mieux avec la version actuelle de JavaScript et de nombreuse entreprises proposent des alternatives à du JavaScript brut. CoffeeScript (Dropbox) et TypeScript (Microsoft) sont des surcouches de JavaScript afin de permettre au développeurs de coder plus efficacement que ne leur permet JavaScript et ainsi répondre aux principales critiques effectuer sur le code JavaScript. Google propose Dart, un concurrent direct de JavaScript, qui n'a jamais réussi à percer. 

Heureusement EcmaScript Harmony, va permettre à JavaScript d'être plus adapté dans la création d'application en JavaScript et la plupart des navigateurs le supporte déjà. Ainsi avec la nouvelle version Harmony orienté application,la question de ce mémoire sur JavaScript, est il uniquement un langage web ne devrait plus être valable officiellement. Pour l'instant au vu des frameworks disponibles la réponse est non mais ce n'est pas une version officielle.

L'avenir de JavaScript dans les 5 prochaines années sera marqué par l'évolution des moteurs JavaScript et aussi par l'activité autour de Node.js coté serveur ce qui en fait un acteur majeur des prochaines années. Mais est-ce une tendance pour les 5 prochaines année à venir comme l'a été Python ou Ruby ou un sérieux changement pour le développement d'application 3-tiers?
\section{EcmaScript}
\label{ch:ecmascript}

\subsection{Présentation}

ECMASCript est un langage de programmation de type script. Il est standardisé par Ecma International dans le cadre de la spécification ECMA-262. Il s’agit d’un standard, donc les spécifications sont mises en œuvre dans différents langages comme JavaScript ou ActionScript.

JavaScript évolue donc en fonction de l'avancement des standard de Ecma-International.

Comme vu précédemment JavaScript a vu le jour en décembre 1995 par Sun et Netscape.

En mars 1996, Netscape implémente le moteur JavaScript dans son navigateur web Netscape Navigator 2.0. Le succès de ce navigateur contribue à l’adoption rapide de JavaScript dans le développement web orienté client. Microsoft qui était à l’époque le seul concurrent a réagit en développant JScript, qu’il inclut ensuite dans Internet Explorer 3.0 en août 1996 pour la sortie de son navigateur.

Netscape soumet alors JavaScript à l’ECMA pour le faire standardiser. Les travaux débutent en novembre 1996, et se terminent en juin 1997 par l’adoption du nouveau standart ECMAScript. Les spécifications sont rédigées dans le document Standard ECMA-262.

ECMAScript existe à ce jour en 5 versions du standart ECMA-262.
\chapter{Ember.js}
\label{ch:ember}

\section*{Introduction}


Ember.js est un framework frontend MVC écrit en JavaScript qui s’éxécute dans le navigateur. Il est destiné aux développeurs qui cherchent à construire des applications web avec autant de puissance et de fonctionalités que les applications natives concurentes.

Ember.js a été créé à partir de concepts introduits par les applications natives telles que Cocoa. Ember.js vous aide à créer une grand expérience pour l’utilisateur. Il vous aidera à organiser toutes les interactions directes qu’un utilisateur peut effectuer sur votre site/applications.

Quand vous croyer que votre code JavaScript va devenir complexe, lorsque le code deviens long et compliquer, lorsque le refactoring du code va être compliquer alors Ember.js vous évite celà. Grave à ça structure MVC (Modèle-vue-controleur) il est facile de faire des modifications ou refactoring de code à n’importe quelle partie de votre code. Il vous permettra également de faire d’adhérer aux principes du DRY (Don’t Repeat Yourself). Le modle associé aux vues et aux controleurs exécute du CRUD (Create, Read, Update, Delete) pour l’informer d’un changement d’état. Il peut envoyer également une demande à la vue de modifier la façon dont la vue représente le modèle actuel. La vue recevera alors des informations du modèle pour créer un rendu graphique. Si vous n’étes pas encore clair sur la façon dont les troies parties interagissent les uns avec les autres, ce qui suit est un simple diagramme illustrant ce principe:

Figure MVC


Ember.js découpe les zones problématiques de votre interface vous permettant de vous concentrer sur une zone à la fois sans le souci d’affecter d’aute parties de votre application. Pour vous donner un exemple de certains domaines de Ember.js, jeter un oeil à la liste suivante:

    Navigation: le routeur d’Ember s’occupe de la navigation dans votre application

    Mise à jour automatique des modèles: La vue d’Ember sont automatiquement mise à jour lorsqu’il y’a une modification. Ce qui signifie qu’ember mettre à jour automatiquement les données sous-jacentes si il y’a un changement.

    Manipulaton des données: Chaque objet créé sera un objet Ember, héritant ainsi de toutes les méthodes Ember.object.

    Comportement asynchrone: Les liaisons et les propriété utilisé avec Ember vous aides à gère l’asynchrone.


Ember.js est plus qu’une bibliothèque. Ember.js prévoit de construire une bonne partie de votre frontend autour de méthodes et d’une architecture carré, créant une solide architecture une fois que vous avez terminé votre application. C’est la principale différence entre Ember et un framework comme Angular.js. Augular s’autorise à être incorporé dans une application existante alors que Ember doit être utilisé avec sa propre architecture et sa philosophie. Backbone.js serait un autre exemple de bibliothèque pouvant être facilement inséré dans des projets JavaScript existants.

Ember.js est un excellent framework pour la gestion complexe des interactions réalisées par les utilisateurs de votre applicatio. Si vous pensez que Ember.js est un framework difficile à apprendre, c’est totalement faux. La seule difficulté pour les développeurs réside dans la compréhension des concepts que Ember.js cherche à mettre en oeuvre. Ember.js favorise convention plutôt que configuration.

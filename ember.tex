\section{Ember}
\label{ch:ember}

\subsection{Présentation}


Ember.js est un framework frontend MVC écrit en JavaScript qui s’éxécute dans le navigateur. Il est destiné aux développeurs qui cherchent à construire des applications web avec autant de puissance et de fonctionnalités que les applications natives concurrentes.

Ember.js a été créé à partir de concepts introduits par les applications natives telles que Cocoa. Ember.js vous aide à créer une grand expérience pour l’utilisateur. Il vous aidera à organiser toutes les interactions directes qu’un utilisateur peut effectuer sur votre site/applications.

Quand vous pensez que votre code JavaScript va devenir complexe, lorsque le code deviens long et compliqué, lorsque le refactoring du code s'annonce complexe alors Ember.js vous permet d'éviter de vous retrouver dans ces situations. Grave à sa structure MVC (Modèle-vue-controleur) il est facile de faire des modifications ou refactoring de code de n’importe quelle partie de votre code. Il vous permettra également d’adhérer aux principes du DRY (Don’t Repeat Yourself). Le modèle associé aux vues et aux contrôleurs exécute du CRUD (Create, Read, Update, Delete) pour l’informer d’un changement d’état. Il peut envoyer également une demande à la vue pour modifier la façon dont la vue représente le modèle actuel. La vue recevra alors des informations du modèle pour créer un rendu graphique.

Ember.js découpe les zones problématiques de votre interface vous permettant de vous concentrer sur une zone à la fois sans le souci d’affecter d’autres parties de votre application. Pour vous donner un exemple de certains domaines de Ember.js, jetez un oeil à la liste suivante:

\begin{itemize}

  \item[\textbullet]
  Navigation: le routeur d’Ember s’occupe de la navigation de l'application

  \item[\textbullet]
  Mise à jour automatique des modèles: Les vues d’Ember sont automatiquement mises à jour lorsqu’il y a une modification. Ce qui signifie qu’ember met à jour automatiquement les données sous-jacentes lorsqu'il y a un changement.

  \item[\textbullet]
  Manipulaton des données: Chaque objet créé sera un objet Ember, héritant ainsi de toutes les méthodes Ember.object.

  \item[\textbullet]
  Comportement asynchrone: Les liaisons et les propriété utilisées avec Ember aident à gérer l’asynchrone.

\end{itemize}

Ember.js est plus qu’une bibliothèque. Ember.js prévoit de construire une bonne partie du frontend autour de méthodes et d’une architecture carrées, créant ainsi une solide architecture une fois que vous avez terminé votre application. C’est la principale différence entre Ember et un framework comme Angular.js. Augular s’autorise à être incorporé dans une application existante alors que Ember doit être utilisé avec sa propre architecture et sa philosophie. Backbone.js serait un autre exemple de bibliothèque pouvant être facilement inséré dans des projets JavaScript existants.

Ember.js est un excellent framework pour la gestion complexe des interactions réalisées par les utilisateurs de votre application. Si vous pensez que Ember.js est un framework difficile à apprendre, c’est totalement faux. La seule difficulté pour les développeurs réside dans la compréhension des concepts que Ember.js cherche à mettre en oeuvre. Ember.js favorise convention plutôt que configuration.

\chapter{Introduction}
\label{ch:introduction}

Le développement JavaScript a nettement changé depuis sa conception. Il est facile d’oublier la mise en œuvre de JavaScript dans le navigateur Netscape, jusqu’au navigateur d’aujourd’hui avec de puissants moteurs, tels que le moteur V8 de Google.

Cela a été un chemin rocailleux impliquant renommage, fusion et normalisation éventuelle comme ECMAScript. Les capacités de JavaScript que nous avons aujourd’hui sont au delà des rêves les plus fous de ses concepteurs.

Malgré son succès et sa popularité, JavaScript est encore largement méconnu. Peu de gens savent que JavaScript est un langage puissant et orienté objet. Ils sont surpris d’en apprendre davantage sur certaines de ses fonctionnalités les plus avancées, telles que l’héritage, les modules et les espaces de noms. Alors, pourquoi JavaScript est il si mal compris?

Un premier élément de réponse s'explique par précédentes implémentations de JavaScript boguées, et le second élément de réponse proviens du nom du préfixe Java suggérant que JavaScript est en quelque sorte lié à Java. En réalité la raison de cette incompréhension est totalement différente. La véritable raison est la façon dont la plupart des développeurs sont initiés à ce langage. Avec d’autres langages, tels que Python ou Ruby, les développeurs font généralement un effort concerté pour apprendre le langage avec l’aide de livres, screencasts et tutoriaux. Jusqu’à récemment, les développeurs recevaient des demandes pour ajouter un peu de validation de formulaire, peut être ajouter un album ou une galerie photo avec du code tout prêt ou un calendrier. Ils utilisaient des scripts qu’ils trouvaient sur Internet, appelé avec peu de compréhension du langage derrière le script. Après cette phase, certains développeurs ajoutent JavaScript à leurs CV.

Récement les moteurs JavaScript et les navigateurs sont devenus si puissant que la construction complète d’applications riches en JavaScript est non seulement faisable mais aussi de plus en plus populaire. Les applications tels que Gmail et Google Maps ont ouvert la voie à une manière complètement différente de penser les applications Web, et les utilisateurs en réclament plus. Les entreprises embauchent pour répondre à la demande des développeurs JavaScript à temps plein. Ce n’est plus un sous-langage relégué à des scripts simples et un peu de validation de formulaire, il est désormais un langage autonome de son propre droit, avec un sérieux potentiel.

Cet afflux de popularité signifie qu’un grand nombre de nouvelles applications JavaScript sont en cours de construction, surtout avec l’émergence de toujours plus de périphériques autonomes comme les smartphones, les tablettes etc

Malheureusement, et peut être en raison de l’histoire de ce langage, beaucoup d’applications JavaScript sont mal conçus. Peu importe la raison, les modèles reconnus et les meilleures pratiques partent à la poubelle. Les développeurs ignorent les modèles architecturaux tels que le Modèle-Vue-Controleur (MVC), mêlant à la place un désordre de HTML et JavaScript à leurs applications .

Mon mémoire va plutôt présenter des bibliothèques, plateformes et frameworks pour structurer et construire vos applications complexe entièrement en JavaScript. Que se soit la partie serveur, les clients légers comme les navigateurs, les clients lourds avec les applications Windows 8 ou Gnome 3, les périphériques mobiles comme les smartphones, tablettes, tv connectés et liseuses ainsi que le stockage des données avec l’utilisation de bases de données. Ce mémoire inclut également la présentation d'outils utilisé pour créer une application de qualité avec l’inclusion des tests, de la documentation ...

Pour ceux qui veule en apprendre plus sur le langage JavaScript de nombreux livres sont disponibles pour apprendre à sa syntaxe et sa structure.


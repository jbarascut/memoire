\section{NPM}
\label{ch:npm}

\subsection{Présentation}

Node Package Manager (NPM) est un gestionnaire de modules pour Node. C'est en quelque sorte un équivalent pour Node de aptitude ou apt-get (les gestionnaires de paquets sous Debian et Ubuntu) avec une interface très similaire. 

NPM permet d’ajouter de nombreuses fonctionnalités à Node (exemple : gestion des sockets avec socket.io, utilisation d’un framework avec Express JS, utilisation de template avec Jade…) en utilisant une commande npm qui s’occupera de télécharger et installer les modules demandés.

Par exemple pour télécharger la dernière version d'Express, il suffit d'exécuter la commande \textit{npm install express} et de laisser l'utilitaire faire son travail. Si quelques mois plus tard, le désire d’obtenir la dernière version d'express se fait sentir, on exécutera simplement \textit{npm update express}.

Cet outil  permet également d’installer directement un exécutable du module grâce a l’option \textit{-g} (très pratique pour Coffee-Script par exemple).

Actuellement le nombre de modules Node.js disponible est supérieur à 30 000 modules.

NPM, ainsi que des informations concernant son installation et son utilisation sont disponibles sur github, comme la grande majorité de l'écosystème node.





\chapter{Client}
\label{ch:client}

\section{Introduction}

Les années 1990 ont été marquées par les architectures client / serveur. Face à la difficulté de déploiement et de maintenance de ces clients lourds sur chaque poste client, les années 2000 ont vu l’explosion de clients dits « légers », qui déportent une grande partie du code sur des serveurs d’applications et réduisent ainsi considérablement les coûts de déploiement. Ces clients légers reposent sur les technologies HTML, JavaScript et CSS.

Avec l'évolution de la demande sur les interfaces web, les applications veulent toujours plus de fonctionnalité dans les navigateurs. Pour palier à ce problème de nombreux frameworks sont apparus avec un niveau d'abstraction plus élevé que les bibliothèques de manipulation du DOM comme jQuery et Prototype. Ces frameworks de haut niveau offrent généralement la gestion de la vue coté client, le routage, la liaison des données, etc. Ceci est utilise, car si vous décidez d'écrire une application modérément complexe en JavaScript avec seulement jQuery, vous devrez mettre eu oeuvre ces fonctionnalités vous-même, ou vous retrouvez avec un désordre d'événements et de rappels.

Pendant longtemps, il n'a eu aucun gagnant clair entre les différents frameworks JavaScript de haut niveau (parfois appelé framework MV*). En 2011 Backbone.js a gagné énormément en popularité. Il a été récemment contestée par Ember.js. Les deux frameworks sont légers, mais ils sont très différents sur les goûts en matière de conception et en philosophies. 

Même si Ember et Backbone attire beaucoup l'attention actuellement, il existe d'autres framewokr avec des approches intéressantes comme Angular et CanJS. Ces framework font sensiblement tous la même choses. Afin de choisir le framework le plus adaptés à vos besoin, il me parait indispensable de voir ensemble leur différences et de les comparés sur des critères communs et indispensable dans pour un développeur.

Une belle façon de comparer les différents frameworks MV* est TodoMVC. TodoMVC  présente l'implémentations d'une applications de gestion de liste de taches dans différents framework MV*. Ainsi il est possible pour un même application de comparer le code source et les différences de plusieurs framework. Le nombre de framework disponible sur TodoMVC étant d'une cinquantaine, j'ai décidé d'en présenter et comparer seulement quatres (Ember, Backbone, Anglular et CanJS).  Ces quatre framework représente pour moi les framework qui ont un fort potentiel et vont être fortement utilisé en entreprise.
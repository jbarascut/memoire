\section{JSON}
\label{ch:json}

\subsection{Présentation}


JSON est un format d’échange de données qui est un sous ensemble de la notation objet littérale en JavaScript. Il a beaucoup gagné en popularité ces derniers temps comme une alternative légère au format XML, en particulier dans les applications AJAX.

Pourquoi ça?

En raison de la capacité de JavaScript pour l’analyse rapide de l’information à l’aide la fonction eval(). JSON ne nécessite pas cependant JavaScript, et il est possible de l’utiliser comme un format d’échange simple pour n’importe quel langage de script.

Voici un exemple de ce que JSON est:

{'détails': {

'id': 1,

'type': 'mémoire',

'auteur': 'Anthony T. Holdener III',

'titre': 'JavaScript, un langage web?',

'detail': {

'pages': 120,

'extra': 22,

'price': {

'eu': 00.00,

'us': 00.00

}

}

}}

Voici à titre de comparaison sont équivalent en XML

<détails id="1" type="mémoire">

<auteur>BARASCUT Jérémy</auteur>

<title>Ajax: The Definitive Guide</title>

<detail>

<pages extra="20">960</pages>

<isbn>0596528388</isbn>

<price us="49.99" ca="49.99" />

</detail>

</details>

Certains développeurs pensent que JSON est une façon plus élégante d’écrire les données. D’autres apprécient sa simplicité. D’autres encore soutiennent qu’il est plus léger.

Peut importe, si JSON est si populaire c’est surtout grace à la technologie AJAX.


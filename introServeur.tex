\chapter{Serveur}

\section{Introduction}
\label{ch:introServeur}



Il y a encore quelques années, l’idée de faire du développement en Javascript faisait hérisser les cheveux de la plupart des programmeurs dont il se serait volontiers passé si une alternative viable existait. Grâce à l'évolution technologique, l'explosion et la démocratisation des RIA (Rich Internet Applications) et la révolution du marché des navigateurs avec Chrome et Firefox, le JavaScript est devenu un langage de première classe dans le paysage web.

A la base JavaScript est également un langage prévue pour tourner coté serveur. Alors pourquoi ne pas développer un serveur d’application avec JavaScript ?

Car malgré ses défauts, Javascript est un langage possédant une expressivité impressionnante, \textbf{orienté objet}, doté de principes hérités des langages fonctionnels (Lisp, etc.) comme la \textbf{closure} ou les \textbf{lambda-functions}, et intégrant le paradigme de \textbf{programmation évènementielle}... Bien loin, d’un langage pour faire de l’accès DOM. 

Ce chapitre a pour but de présenter le développement coté serveur avec  node.js, une technologie basée sur V8, le moteur JavaScript de Google Chrome, dont elle étend les capacités en ajoutant de nombreuses fonctionnalités d'I/O (Entrées/Sorties).

Node est tellement important et performant qu’un mémoire/livre entier n’arriverait pas à couvrir son ensemble.

Ce chapitre va présenter \textbf{Node.js}, son gestionnaire de paquet \textbf{NPM} ainsi que deux framework assez prometteur \textbf{Express} et \textbf{Meteor} permettant de simplifier le développement d’application serveur avec Node.

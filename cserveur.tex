\section{Conclusion}
\label{ch:conclusion serveur}

En peu de temps, c’est à dire 2009, Node.js a suscité l'intérêt de milliers de développeurs expérimentés et possède actuellement un gestionnaire de paquet avec de nombreux modules intéressant. Node.js est également une raison de la création de nombreuse startup à travers le monde.

J’ai essayé au travers de ce chapitre de vous présenter la petite révolution de Node.js dans le développement d’application 3-tiers et de son intéret.

Node.js a beaucoup de points forts qui peuvent faire de lui une option sérieuse pour le développement d'applications coté serveur. Comme nous l’avons déjà vu, il tourne avec V8, le moteur javascript de chez google, ce qui lui confère de très bonnes performances. Ensuite, il utilise une gestion événementielle des I/O, non-bloquante. EventMachine, Twisted ou encore Tornado font de même en Ruby ou Python, mais Node.js pousse le concept plus loin (les accès aux fichiers sont asynchrones, par exemple), et la syntaxe javascript s'y prête particulièrement bien. C'est vraiment un élément clé pour la scalabilité. Enfin, Node.js utilise maintenant les modules de CommonJS, favorisant ainsi la réutilisation avec les autres frameworks JS.

Bref, Node.js a un gros potentiel, et la prolifération des frameworks (Picard, fab, Express, Djangode, Vroom, Node-router, Bomber…) et bibliothèques (tmpl-node, node-restclient, restclient, node-crypto, node-compress, cookie-node…) laisse penser que Node.js a un brillant avenir devant lui.

\subsection{Wordsquared.com}

Ce dernier point s'adressera particulièrement aux sceptiques, pour lesquels node n'est encore qu'un gadget. Le site wordsquared.com est l'un des quelques sites basés sur node aujourd'hui. Wordsquared est particulier, car il s'agit d'une démonstration technologique et technique d'une rare qualité : il s'agit en fait d'un jeu de scrabble géant, multijoueur, en temps réel. La réalisation de l'application est impressionnante, tout comme sa réactivité. C'est sans aucun doute la meilleure preuve de maturité du serveur node.js/
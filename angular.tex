\chapter{Angular}
\label{ch:angular}

\section*{Introduction}

AngularJs est un framework structurel pour les applications web dynamiques. Il est écrit en  JavaScript et est sous Licence MIT.

C’est un framework open-source au même titre que MooTools, Prototype, Dojo ou Jquery.

AngularJs est développé par Google et sa communauté.

Angular permet d’utilise le HTML comme langage de template et vous permet d’étendre la syntaxe la syntaxe du HTML pour exprimer les composants de votre application claire et succincte.

Il a pour but de simplifier l’écrire du JavaScript en simplifiant la syntaxe et en comblant les faiblesses de ce langage en lui ajoutant de nouvelles fonctionalités. Le but d’angularJs est de faciliter la réalisation d’applications web monopages.

AngularJs peut être utiliser avec ou sans Jquery pour la manipulation du DOM.

Le framewokr adapte et étend le HTML taditionnel pour servir le contenu dynamique de façon améliorée grâce à un data-binding bidirectionnel qui permet la synchronisation automatique des modèles et des vues. En conséquence AngularJS minore l’importance des manipulations DOM et améliore la testabilité du code.

AngularJs essaie d’être une solution complête pour créer une application web. Il est livré avec tout ce que vous avez besoin pour construire une application CRUD de façon cohérente: liaison des données, directives de bases pour les templates, la validation du formulaire, le routage, le deep-linking, la réutilisation de composants et l’injection des dépendances. AngularJs permet l’écriture de tests unitaire, cas de test, mocks etc.

Objectifs de conception du framework:

    Découpler les manipulations du DOM de la logique métier. Cela améliore la testabilité du code.

    Guider les développeurs pendant toute la durée du périple de la construction d'une application: de la conception de l'interface utilisateur, en passant par l'écriture de la logique métier, jusqu'au test de l'application.

    Considérer le test d'une application aussi important que l'écriture de l'application elle-même. La difficulté de la phase de test est considérablement impactée par la façon dont le code est structuré.

    Découpler les côtés client et serveur d'une application. Cela permet au développement logiciel des côtés client et serveur de progresser en parallèle, et permet la réutilisabilité de chacun des côtés.

    Rendre les tâches faciles évidentes et les tâches difficiles possibles.


Angular est un framework MVC et encourage le couplage faible entre la présentation, les données, et les composants métiers. En utilisant l’injection de dépendances, Angular apporte aux applications web coté client les services traditionnellement apportés coté serveur, comme les contrôleurs de vues. En conséquence, une bonne partie du fardeau supporté par le back-end est supprimée, ce qui conduit à des applications web beaucoup plus légères et maintenables.

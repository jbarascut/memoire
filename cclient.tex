\section{Conclusion}
\label{ch:conclusion client}

Au vue de la présentation des différents framework client, il peut être extrêmement difficile de choisir un framework MVC JavaScript pour développer notre client web. Il y’a tellement de facteurs à considérer et tellement d’options à étudier que la sélection d’un framework peut être hardue. Pour avoir une idée de toutes les alternatives possibles, il y a le site TodoMVC qui permet de comparer l’utilisation de plusieurs framework codés afin de réaliser une liste de choses à faire (TODO).

Je vous ai présenté 4 frameworks: Angular, BackBone, CanJS et Ember. J’ai donc décidé de réaliser ce mémoire autour de la comparaison de ces derniers afin de faciliter le choix du framework à utiliser. Je vais passer en revue plusieurs facteurs que vous pourriez envisager pour en choisir un.

Pour chaque facteur, j’ai attribué une note comprise entre 1 et 5. Où 1 est pauvre et 5 est riche. J’ai essayé d’être impartial dans mon mémoire, mais bien entendu mon objectivité est fortement compromise parce que mes sources sont uniquement basées sur les exemples du site TodoMVC.

%Inclure logo image framework

\subsection{Fonctionnalités}

Il y’a des éléments vraiment importants qu’un framework doit fournir pour avoir les bases nécéssaires à la réalisation d’applications utiles. Fait-elle une liaison entre le modèle et la vue? une liaison bidirectionnelle entre le modèle et la vue? l’utilisation de filtres? les propriétés sont elles calculées? Y a-t-il une validation des formulaires? etc. Cette liste peut être une très longue. Voici une comparaison de ce que je considère comme des caractéristiques vraiment importantes dans un framework MVC.

\begin{table}
\begin{center}
\begin{tabular}{|l|c|c|c|c|}
  \hline 
  Fonctionnalités & Angular & Backbone & CanJS & Ember \\
  \hline 
  Observables & O & O & O & O \\
  \hline 
  Routage & O & O & O & O \\
  \hline 
  Liaison modèle-vue & O & - & O & O \\
  \hline 
  Liaison modèle-vue bidirectionnelle & O & - & - & O \\
  \hline 
  Vue partielle & O & - & O & O \\
  \hline 
  Vue filtrée & O & - & O & O \\
  \hline
\end{tabular}
\end{center}
\caption{Mon tableau}
\end{table}

\textbf{Observables:} Objets qui peuvent être observés lors des changements.

\textbf{Routage:}Ecoute les changements d’URL du navigateur et permet d’agir en conséquence.

\textbf{Liaison modèle-vue:}Utilisation d’objets observables dans les vues, les vues ayant un rafraîchissement automatique lorsque le changement de l’objet est observable.

\textbf{Liaison bidirectionnelle entre le modèle et la vue:}Permet à la vue de pousser la modification de l’objet observation automatiquement, par exemple, un formulaire de saisie.

\textbf{Vue partielle:} Vues qui comprennent d’autres vues

\textbf{Vue filtrée:} Avec des vues qui affichent des objets filtrés par un cetain nombre de critères.

\subsubsection{Résultat}

Sur la base de ces caractéristiques les scores sont les suivants:

\begin{tabular}{|c|c|c|c|}
  \hline 
  Angular & Backbone & CanJS & Ember \\
  \hline 
  5 & 2 & 4 & 5 \\
  \hline
\end{tabular}

Il est important de noter que Backbone peut faire la plupart de ces choses avec beaucoup d’écriture de code manuel ou à l’aide de plug-in. Mais je n’ai considéré que les fonctions disponibles de base dans les framework.

\subsection{Flexibilité}

Il existe des centaines de plugin et bibliothèques qui ont actions bien spécifiques. Ils le font habituellement mieux que le framework d'origine. Il est donc important d’être en mesure de pouvoir intégrer ces bibliothèques dans le framework MVC choisi.

\textbf{Backbone} est le framework le plus souple, car il est celui qui à le moins de conventions et de normes. Vous êtes obligés de prendre beaucoup de décisions lors du développement.

\textbf{CanJS} est presque aussi souple que Backbone car il permet d’intégrer facilement d’autres bibliothèques avec un minimum d’effort. Avec CanJS il est possible également d’utiliser un moteur de rendu totalement différent. Par exemple l’utilisation du moteur de rendu "Rivets" ne pose aucun problème. Mais il est vivement recommandé d’utiliser le moteur de rendu du framework originel.

\textbf{Ember} et \textbf{Angular} sont aussi des framework souples dans une certaine mesure, mais il se peut que l’utilisation d’autres biliothèques soit difficile à mettre en place en complément du framework originel.


\subsubsection{Résultat}
\begin{tabular}{|c|c|c|c|}
  \hline 
  Angular & Backbone & CanJS & Ember \\
  \hline 
  3 & 5 & 4 & 3 \\
  \hline
\end{tabular}


\subsection{La documentation et la courbe d'apprentissage}

\subsubsection{Angular}

Angular est un framework qui lorsqu’on le découvre procure un effet “wow”. Il peut faire des choses étonnantes, comme les liaisons bidirectionnelles, sans avoir à apprendre beaucoup de choses. Il semble assez facile à prendre en main d'un premier abord. Mais une fois les bases acquises, la courbe d’apprentissage se met alors à stagner. Il s’agit d’un framework complexe avec beaucoup de particularités. La lecture de la documentation n’est pas facile car il y’a beaucoup de jargon spécifique à Angular et un manque important d’exemple.

\subsubsection{Backbone}

Les bases de Backbone sont elles aussi assez faciles à assimiler. Mais encore une fois, une fois dépassé le stade des bases, on s'aperçoit que la documentation ne donne pas assez de conseils sur la façon de structurer son code. Il devient alors obligatoire de regarder ou lire des tutoriels pour apprendre certaines des meilleures pratiques de backbone. Il se peut également que l’apprentissage d’une autre bibliothèque soit obligatoire pour faire avancer votre développement (par exemple Marionette ou Thorax). Au final, l’apprentissage de Backbone n’est pas plus facile.

\subsubsection{CanJS} 

CanJS en comparaison est le plus facile à apprendre. Juste en lisant le site internet du framework (http://canjs.us), l’essentiel de ce que l’on a besoin de savoir est présenté pour être productif. Il y’a bien sûr encore beaucoup à apprendre, mais la nécessité de recourir à l’aide survient à de rares occasions (tutoriels, forum, irc).

\subsubsection{Ember}

Ember a aussi une courbe d’aprentissage abrupte comme Angular, mais l’apprentissage de Ember est un peut plus facile que Angular. Il nécessite par contre un investissement important au début pour acquérir les bases. Angular en revanche permet de faire des choses incroyables sans en avoir trop à apprendre. Ember manque de cet effet “wow”.


\subsubsection{Résultat}
\begin{tabular}{|c|c|c|c|}
  \hline 
  Angular & Backbone & CanJS & Ember \\
  \hline 
  2 & 4 & 5 & 3 \\
  \hline
\end{tabular}


\subsection{La productivité des développeurs}

Après avoirr appris correctement le framework ce qui importe le plus est de savoir comment le framework permet d’avoir une productivité performante en mixant habilement conventions, magie et conception la plus rapide possible.

\subsubsection{Angular}

Une fois que l’on maîtrise Angular correctement, on devient très productif. Il n'y a aucun doute à ce sujet. Il n’a cependant pas le score le plus élevé parce que Ember franchi une étape supplémentaire dans cette catégorie.


\subsubsection{Backbone}


Backbone oblige à écrire beaucoup de code réutilisable, ce qui est pour certains projet totalement inutile. Certain développeur pensent qu'il s'agit davantage d'une contrainte importante allant à l’encontre de la productivité des développeurs.

\subsubsection{CanJS}

CanJS ne brille pas mais ne déçoit pas non plus dans ce domaine. En raison de la faible courbe d’apprentissage vous pouvez devenir très productif très tôt avec ce framework.


\subsubsection{Ember}

Ember brille particulièrement ici. Parce qu’il est plein de conventions, il fait beaucoup de choses à votre place. Tout ce qu’il reste à faire c'est d’apprendre et d'appliquer ces conventions et Ember s’occupe du reste.




\subsubsection{Résultat}
\begin{tabular}{|c|c|c|c|}
  \hline 
  Angular & Backbone & CanJS & Ember \\
  \hline 
  4 & 2 & 4 & 5 \\
  \hline
\end{tabular}

\subsection{Communauté}

Est-il facile de trouver de l’aide, des tutoriels et des experts?

La communauté \textbf{Backbone} est énorme, c'est indubitable. On trouve des dizaines de tutoriaux sur Backbone et une communauté très active sur StackOverflow et IRC.

Les communautés \textbf{Angular} et \textbf{Ember} sont également assez grandes. Il y a également de nombreux tutoriel ainsi qu'une forte activité sur StackOverflow et IRC, mais pas autant que sur Backbone.

La communauté \textbf{CanJS} est assez faible en comparaison, mais heureusement, très active et utile. La taille de la communauté n’est par conséquent pas un frein important.

\subsubsection{Résultat}
\begin{tabular}{|c|c|c|c|}
  \hline 
  Angular & Backbone & CanJS & Ember \\
  \hline 
  4 & 5 & 3 & 4 \\
  \hline
\end{tabular}

\subsection{Ecosystème}

Y’a t’il un écosystème de plugin et de bibliothèques?

Là encore \textbf{Backbone} surpasse les autres. Il y a un volume très important de plugins disponible. L’écosystème d’\textbf{Angular} devient assez intéressant avec entre autre l’interface utilisateur. L’écosystème d’\textbf{Ember} est moins développé mais il devrait s’améliorer en raison de sa popularité importante.\textbf{CanJS} en comparaison dispose du plus petit écosystème.



\subsubsection{Résultat}
\begin{tabular}{|c|c|c|c|}
  \hline 
  Angular & Backbone & CanJS & Ember \\
  \hline 
  4 & 5 & 2 & 4 \\
  \hline
\end{tabular}

\subsection{Poid}

Cela peut être un facteur important, surtout pour du développement mobile.

\subsubsection{Poid des frameworks seul (sans aucune dépendances)}


\begin{tabular}{|c|c|c|c|}
  \hline 
  Angular & Backbone & CanJS & Ember \\
  \hline 
  80ko & 61ko & 57ko & 269ko \\
  \hline
\end{tabular}

Backbone est le plus léger et les développeurs soulignent souvent ce fait. Mais ce n’est pas le cas si l’on compte les dépendances.

\subsubsection{Poids des frameworks avec dépendances}

\textbf{Angular} est le seul framework du groupe qui ne nécessite pas de bibliothèques supplémentaires pour travailler.

Tout les autres ont besoins de dépendances pour fonctionner.

\textbf{Backbone} a au moins besoin de Underscore et Zepto. Il est possible d’utiliser les mini-templates de Underscore pour le rendu des vues, mais la plupart du temps, l’utilisation d’un véritable moteur de template est appréciable comme Moustache par exemple. Au final le poids est de 61ko.

\textbf{CanJS} a besoin d’au moins Zepto. 57ko.

\textbf{Ember} a besoin de jQuery et de Handlebars. 269ko.

\subsubsection{Résultat}
\begin{tabular}{|c|c|c|c|}
  \hline 
  Angular & Backbone & CanJS & Ember \\
  \hline 
  4 & 5 & 5 & 2 \\
  \hline
\end{tabular}


\subsection{Performance}

La performance n’est pas forcément un facteur déterminant dans le choix d’un framework car ils sont tous assez performant pour la plupart des cas d’utilisation. Mais cela dépend bien sur de l’utilisation du framework. S'il est utilisé pour la création d’un jeu alors les performances deviennent importantes.

Il y a sur internet de nombreux tests de performances avec ces frameworks. Par contre la fiabilité de ces tests n'est pas une valeur sure car il est impossible de vérifier si ces tests ont été réalisés de dans bonnes conditions et avec un code de qualité.

Cependant, au vu des tests, il semblerait que \textbf{CanJS} ait un avantage lorsqu'il s’agit de performances, en particulier dans l’affichage de la vue. De plus, \textbf{Angular} est le moins performant.

\subsubsection{Résultat}
\begin{tabular}{|c|c|c|c|}
  \hline 
  Angular & Backbone & CanJS & Ember \\
  \hline 
  3 & 4 & 5 & 4 \\
  \hline
\end{tabular}


\subsection{Maturité}

Est-ce un framework mature, est-ce prouvé en production, est-ce que de nombreuses applications web l’utilise?

\textbf{Backbone} a de nombreux sites construit avec lui. Sa base de code n’a pas eu de grands changements dans les deux dernières années ce qui est une bonne chose du point de vue de la maturité.

Bien que \textbf{Ember} ne soit pas nouveau, il a subi de profonds changements lors de sa progression, transformant tout pour atteindre une forme stable dans les trois derniers mois. Par conséquent il ne s’agit pas d’un framework mature.

\textbf{Angular} semble plus stable et éprouvé que Ember mais moins que Backbone.

\textbf{CanJS} est une extraction de JavaScriptMVC, une bibliothèque qui a été lancé en 2008 et possède un important retour d’expériences et d’application construites.


\subsubsection{Résultat}
\begin{tabular}{|c|c|c|c|}
  \hline 
  Angular & Backbone & CanJS & Ember \\
  \hline 
  4 & 5 & 4 & 3 \\
  \hline
\end{tabular}



\subsection{Sécurité des fuites mémoire}

Il s’agit d’une considération importante si l’on construit des applications en single page, qui sont destinées à rester ouvert pendant de longues périodes. Si l’on ne veut pas que l’application soit une fuite mémoire, cela peut devenir un réel problème. Malheureusement les fuites peuvent se faire assez facilement, surtout si l’on crée soit même des écoutes pour les évènements DOM.

\textbf{Angular}, \textbf{CanJS} et \textbf{Ember} traiteront de façon efficace et aussi longtemps les fuites mémoire du moment que l’on suit les meilleures pratiques. \textbf{Backbone} en revanche oblige à faire ce travail manuellement dans une méthode de désallocation mémoire.



\subsubsection{Résultat}
\begin{tabular}{|c|c|c|c|}
  \hline 
  Angular & Backbone & CanJS & Ember \\
  \hline 
  5 & 3 & 5 & 5 \\
  \hline
\end{tabular}


\subsection{Testabilité}

Est-il facile de tester le code?

Les clés pour avoir un grand code testable sont la modularité (avoir de petits morceaux qui peuvent être testés séparément) et l’injection de dépendances (être capable de changer les dépendances dans les tests).

Il est possible de le faire avec la plupart des frameworks du moment que l’on utilise les bons modèles, mais ce n’est pas simple et cela oblige de changer ses habitudes d’applications.

La modularité et l’injection de dépendances sont des caractéristiques fortes de \textbf{Angular}, en décourageant activement de faire les choses d’une autre manière que Angular l'a prévue. Cela conduit généralement à un code plus facile à tester. Pour cette raison, Angular dispose d'un avantage dans ce domaine.


\subsubsection{Résultat}
\begin{tabular}{|c|c|c|c|}
  \hline 
  Angular & Backbone & CanJS & Ember \\
  \hline 
  5 & 4 & 4 & 4 \\
  \hline
\end{tabular}


\subsection{Goût personnel}

C’est probablement l’un des facteurs les plus importants lors du choix d’un framework.
Mais il n’est pas possible de noter ce sujet et cela reste à l’entière appréciation du développeur.

\subsection{Total}

En rassemblant les résultats on arrive à un palmarès. Ceci n’est qu’une opinion forgée à l'aide de TodoMVC ainsi qu'à la lecture de différents sites internet. Il ne représente pas un résultat fiable et sert simplement à dégager un point de vue.


\begin{tabular}{|l|c|c|c|c|}
  \hline 
  Total général & Angular & Backbone & CanJS & Ember \\
  \hline 
  Fonctionnalités & 5 & 2 & 4 & 4 \\
  \hline 
  Flexibilité & 3 & 5 & 4 & 3 \\
  \hline 
  La documentation et la courbe d'apprentissage & 2 & 4 & 5 & 3 \\
  \hline 
  La productivité des développeurs & 4 & 2 & 4 & 5 \\
  \hline 
  Communauté & 4 & 5 & 3 & 4 \\
  \hline 
  Écosystème & 4 & 5 & 2 & 4 \\
  \hline
  Poid & 4 & 5 & 5 & 2 \\
  \hline
  Performance & 3 & 4 & 5 & 4 \\
  \hline
  Maturité & 4 & 5 & 4 & 3 \\
  \hline
  Sécurité des fuites mémoire & 5 & 3 & 5 & 5 \\
  \hline
  Testabilité & 5 & 4 & 4 & 4 \\
  \hline
  Total & 43 & 44 & 45 & 42 \\
  \hline
\end{tabular}

Si le développeur accorde autant de poid à chaque facteur, il s’agit d’un concours serré, ll n’y a ni gagnants ni perdants. Donc je suppose que tout revient au goût personnel du développeur ou à l’affection de poids différents pour chaque catégories.